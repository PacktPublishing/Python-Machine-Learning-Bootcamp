\documentclass[USenglish,final,authoryear,12pt]{article}



\usepackage{amsmath}
\usepackage[a4paper]{geometry}
 \geometry{
 	a4paper,
 	total={210mm,297mm},
 	left=20mm,
 	right=20mm,
 	top=20mm,
 	bottom=20mm,
 }

\usepackage{sectsty}
\usepackage{cancel} %need to use this
\usepackage{babel}
%\allsectionsfont{\normalfont\sffamily\bfseries}
\usepackage{graphicx}
\usepackage{caption}
\usepackage{subcaption}
\usepackage{longtable}
%\begin{figure}[h]
%	\centering
%	\includegraphics[scale=0.5]{Image1.png}
%	\caption{Size of observable universe comparison between standard model of cosmology with inflation and without inflation [1].}
%\end{figure}

\begin{document}
\begin{enumerate}
	\item Using the petal length and petal width feature of the iris dataset, visualize the results of a fitted decision tree of max\_depth=2 trained on these two features (you can use the plot decision boundary function from the lecture).
	\item Rotate the petal length and petal width features by -15$^\circ$, re-train the decision tree on this rotated dataset, and visualize the decision tree results again.\newline
	\textbf{Hint:} To rotate a point (x, y) by $\theta^\circ$ the new points are $(x\cos\theta - y\sin\theta, x\sin\theta + y\cos\theta)$\newline
	\textbf{Hint:} Usually the cos and sin functions use radians, to convert from degree to radians you can multiply the degree value by $\frac{\pi}{180}$
\end{enumerate}
\end{document}

