\documentclass[USenglish,final,authoryear,12pt]{article}



\usepackage{amsmath}
\usepackage[a4paper]{geometry}
 \geometry{
 	a4paper,
 	total={210mm,297mm},
 	left=20mm,
 	right=20mm,
 	top=20mm,
 	bottom=20mm,
 }

\usepackage{sectsty}
\usepackage{cancel} %need to use this
\usepackage{babel}
%\allsectionsfont{\normalfont\sffamily\bfseries}
\usepackage{graphicx}
\usepackage{caption}
\usepackage{subcaption}
\usepackage{longtable}
%\begin{figure}[h]
%	\centering
%	\includegraphics[scale=0.5]{Image1.png}
%	\caption{Size of observable universe comparison between standard model of cosmology with inflation and without inflation [1].}
%\end{figure}

\begin{document}
\begin{enumerate}
	\item Play around with the make moons and make circles dataset, adding in more or less points, increasing/decreasing noise, changing the fraction, and also take your Kernel PCA and tune the hyperparameters by different values (e.g. gamma 0.1, 1, 10, 20), see how the decomposition looks like in each of these cases and the effects that the hyperparameters have.
\end{enumerate}
\end{document}

