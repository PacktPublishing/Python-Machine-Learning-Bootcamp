\documentclass[USenglish,final,authoryear,12pt]{article}



\usepackage{amsmath}
\usepackage[a4paper]{geometry}
 \geometry{
 	a4paper,
 	total={210mm,297mm},
 	left=20mm,
 	right=20mm,
 	top=20mm,
 	bottom=20mm,
 }

\usepackage{sectsty}
\usepackage{cancel} %need to use this
\usepackage{babel}
%\allsectionsfont{\normalfont\sffamily\bfseries}
\usepackage{graphicx}
\usepackage{caption}
\usepackage{subcaption}
\usepackage{longtable}
%\begin{figure}[h]
%	\centering
%	\includegraphics[scale=0.5]{Image1.png}
%	\caption{Size of observable universe comparison between standard model of cosmology with inflation and without inflation [1].}
%\end{figure}

\begin{document}
\begin{enumerate}
	\item Train KMeans clustering on the MNIST dataset. Then propagate the original label (the one we'd get through manual assignment or, in this case, from the know true values) of each cluster center to the x\% closest datapoints to the cluster center. Then train a classifier on this reduced dataset and evaluate its performance against an unseen test or validation set.
\end{enumerate}
\end{document}

